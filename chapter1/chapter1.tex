El presente escrito es la primera versión  de mis apuntes personales que podría ser una propuesta de libro con el título ``Introducción a la Termodinámica''. Si bien espero escribir aplicaciones varias, no aparecerán del todo en esta  primera versión. La intensión es exponer las nociones fundamentales de la termodinámica enfocada a  estudiantes de astronomía y (eventualmente) de física. Espero escribir un PREFACIO tan pronto haya terminado el escrito (osea, lo escribiré de último aunque en la obra inicial va después del contenido o índice). Luego en el primer capítulo pretenderé hacer una pincelada histórica, iniciando con el desarrollo de las mediciones de volumen, presión y temperatura. También, alguna nota histórica breve sobre las máquina térmicas, pues ellas impulsaron la creación de la termodinámica como teoría física. Las leyes de los gases, y otros aspectos interesantes como la formulación de las dos primeras leyes y por ende la introducción del concepto de calor. Si no se extiende mucho este primer capítulo, haría una comparación con formulaciones alternativas, pues en esencia seguiré la formulaciones de Carathéodory.